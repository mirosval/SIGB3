\pagebreak
\section{Introduction}

Computer graphics and image analysis are two very large fields in computer science that concern themselves with capturing, creation, processing and displaying data in form of images. In this report we are describing some of the techniques used in these two fields and the process we have gone through to understand how they work and how they can be applied to solve a variety of tasks.

The following is a compilation of three assignments in the \textit{Introduction to Graphics and Image Analysis} course. Each assignment is a separate project, and even though assignments 2 and 3 share some of the core ideas, they are presented here somewhat independently of each other. Throughout the report, we have used Python and OpenCV as our main tools.

In the first part we look at the challenges we encountered while building an eye tracking software. We were tasked with detecting the pupil, iris and glints in an image of an eye. We have used techniques such as morphologic operations on binary images, blob detection or gradient images.

The second part was dedicated to homographies and projections. We have looked at how to work with virtual cameras, how to use a projection matrix, what is a homography, how to perform texture mapping and how to create augmented images.

The last part presented to us challenges related to 3D to 2D mapping, projection, texturing a 3D object, and calculating shading using the Phong shading model.