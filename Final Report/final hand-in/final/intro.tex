\pagebreak
\section{Introduction}

We were tasked with the construction of a software eye tracker. In the following text we describe our approach to this challenge, the methods we have used, and the results we have obtained.  We have built the eye tracker using Python and OpenCV as a toolkit. The purpose of the eye tracker is to correctly and accurately detect the eye (pupil, iris) and glints on the eye in every image where there is an eye and detect nothing if there is no eye present in the image.

In general, eye tracker algorithms are reducing a huge amount of data (typical 640x480 RGB image is almost 1MB of uncompressed data) to just a few values representing the location and radius of pupil, iris and various glints. Therefore there must be many ways of performing this reduction, some better than others. Our goal in this report is to describe the approach we have taken to solving this problem, which algorithms we have used and what observations and results we have achieved.

As with any work, compromises have to be made and the time available for us to solve this problem is limited. Therefore at some point we must stop looking for more advanced improvements, even though this process could theoretically go on forever. In this report we present the results we have achieved and interesting observations we have made during the process and evaluate how our proposed solution performed.

We wanted to make the tracker as robust as possible, without the need for per-sequence parametrization. It should perform reasonably well without needing any parameter tweaking. Therefore most our efforts have been targeted at eliminating the parametrization in as many places as possible.

The hardest matter that we have faced has been to build the tracker as solid as possible, without the need for per-sequence parameterization. It should work in all circumstances as good as possible. This is of course especially hard.
