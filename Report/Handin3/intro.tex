\pagebreak{}
\section{Introduction}

This assignment has been divided to two main parts as augmented a cube and shading it. In the first part of the assignment we were tasked to learn more about the mathematics behind the camera calibration and computer graphics. Like the two previous assignment, to obtain our goal we used Python and OpenCV as a toolkit.

The purpose for the first part of the assignment was to be able to use the camera matrix to map the points from a coordinate system to another in a hierarchy of the coordinate systems (world, camera, object) and for the second part it was to be able to implement the illumination and shading on the cube that are the two basic concepts in computer graphics. 

In this assignment we covered some uses of camera matrix and augmented reality. Our result for the first part of the assignment was the augmentation on a pattern plane in front of the camera. We mapped a cube with a texture map that can move with the plane movement and seems like it has been attached to the plane. For the second part of our assignment the result was a shaded augmentation to have the cube closer to the reality.
